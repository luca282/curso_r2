\documentclass[]{article}
\usepackage{lmodern}
\usepackage{amssymb,amsmath}
\usepackage{ifxetex,ifluatex}
\usepackage{fixltx2e} % provides \textsubscript
\ifnum 0\ifxetex 1\fi\ifluatex 1\fi=0 % if pdftex
  \usepackage[T1]{fontenc}
  \usepackage[utf8]{inputenc}
\else % if luatex or xelatex
  \ifxetex
    \usepackage{mathspec}
  \else
    \usepackage{fontspec}
  \fi
  \defaultfontfeatures{Ligatures=TeX,Scale=MatchLowercase}
\fi
% use upquote if available, for straight quotes in verbatim environments
\IfFileExists{upquote.sty}{\usepackage{upquote}}{}
% use microtype if available
\IfFileExists{microtype.sty}{%
\usepackage{microtype}
\UseMicrotypeSet[protrusion]{basicmath} % disable protrusion for tt fonts
}{}
\usepackage[margin=1in]{geometry}
\usepackage{hyperref}
\hypersetup{unicode=true,
            pdftitle={PDAD 2018},
            pdfauthor={Luca Fonseca},
            pdfborder={0 0 0},
            breaklinks=true}
\urlstyle{same}  % don't use monospace font for urls
\usepackage{graphicx,grffile}
\makeatletter
\def\maxwidth{\ifdim\Gin@nat@width>\linewidth\linewidth\else\Gin@nat@width\fi}
\def\maxheight{\ifdim\Gin@nat@height>\textheight\textheight\else\Gin@nat@height\fi}
\makeatother
% Scale images if necessary, so that they will not overflow the page
% margins by default, and it is still possible to overwrite the defaults
% using explicit options in \includegraphics[width, height, ...]{}
\setkeys{Gin}{width=\maxwidth,height=\maxheight,keepaspectratio}
\IfFileExists{parskip.sty}{%
\usepackage{parskip}
}{% else
\setlength{\parindent}{0pt}
\setlength{\parskip}{6pt plus 2pt minus 1pt}
}
\setlength{\emergencystretch}{3em}  % prevent overfull lines
\providecommand{\tightlist}{%
  \setlength{\itemsep}{0pt}\setlength{\parskip}{0pt}}
\setcounter{secnumdepth}{0}
% Redefines (sub)paragraphs to behave more like sections
\ifx\paragraph\undefined\else
\let\oldparagraph\paragraph
\renewcommand{\paragraph}[1]{\oldparagraph{#1}\mbox{}}
\fi
\ifx\subparagraph\undefined\else
\let\oldsubparagraph\subparagraph
\renewcommand{\subparagraph}[1]{\oldsubparagraph{#1}\mbox{}}
\fi

%%% Use protect on footnotes to avoid problems with footnotes in titles
\let\rmarkdownfootnote\footnote%
\def\footnote{\protect\rmarkdownfootnote}

%%% Change title format to be more compact
\usepackage{titling}

% Create subtitle command for use in maketitle
\providecommand{\subtitle}[1]{
  \posttitle{
    \begin{center}\large#1\end{center}
    }
}

\setlength{\droptitle}{-2em}

  \title{PDAD 2018}
    \pretitle{\vspace{\droptitle}\centering\huge}
  \posttitle{\par}
    \author{Luca Fonseca}
    \preauthor{\centering\large\emph}
  \postauthor{\par}
      \predate{\centering\large\emph}
  \postdate{\par}
    \date{2019-07-30}

\usepackage[brazilian]{babel}
\usepackage[utf8]{inputenc}
\usepackage{float}

\begin{document}
\maketitle

\section{PDAD 2018}\label{pdad-2018}

A PDAD 2018 visitou \textbf{21.908} domicílios e coletou informações de
69.654 moradores.

\subsection{Estrutura etária}\label{estrutura-etaria}

A estrutura etária dos moradores do Distrito Federal é apresentada na
Figura \ref{fig:piramide}.

\begin{figure}[H]
\includegraphics{relatório_files/figure-latex/piramide-1} \caption{Pirâmide etária \label{fig:piramide}}\label{fig:piramide}
\end{figure}

Os valores específicos podem ser verificados na tabela
\ref{tab:piramide}.

\begin{table}[!hb]
\centering
\begin{tabular}{llrrr}
  \hline
idade\_faixas & sexo & n & n\_low & n\_upp \\ 
  \hline
0 a 4 anos & Feminino & 96.993,00 & 90.751,18 & 103.234,82 \\ 
  0 a 4 anos & Masculino & 102.054,00 & 96.298,94 & 107.809,06 \\ 
  5 a 9 anos & Feminino & 88.139,00 & 84.280,28 & 91.997,72 \\ 
  5 a 9 anos & Masculino & 92.894,00 & 87.908,00 & 97.880,00 \\ 
  10 a 14 anos & Feminino & 103.596,00 & 97.016,81 & 110.175,19 \\ 
  10 a 14 anos & Masculino & 107.671,00 & 101.371,70 & 113.970,30 \\ 
  15 a 19 anos & Feminino & 114.842,00 & 108.552,76 & 121.131,24 \\ 
  15 a 19 anos & Masculino & 117.348,00 & 113.302,15 & 121.393,85 \\ 
  20 a 24 anos & Feminino & 121.383,00 & 116.077,67 & 126.688,33 \\ 
  20 a 24 anos & Masculino & 117.944,00 & 112.684,75 & 123.203,25 \\ 
  25 a 29 anos & Feminino & 126.544,00 & 120.745,52 & 132.342,48 \\ 
  25 a 29 anos & Masculino & 119.316,00 & 114.827,44 & 123.804,56 \\ 
  30 a 34 anos & Feminino & 137.590,00 & 130.611,01 & 144.568,99 \\ 
  30 a 34 anos & Masculino & 126.393,00 & 119.632,17 & 133.153,83 \\ 
  35 a 39 anos & Feminino & 142.423,00 & 136.611,53 & 148.234,47 \\ 
  35 a 39 anos & Masculino & 125.263,00 & 119.152,44 & 131.373,56 \\ 
  40 a 44 anos & Feminino & 125.699,00 & 120.103,34 & 131.294,66 \\ 
  40 a 44 anos & Masculino & 109.759,00 & 104.143,37 & 115.374,63 \\ 
  45 a 49 anos & Feminino & 104.923,00 & 100.152,08 & 109.693,92 \\ 
  45 a 49 anos & Masculino & 92.130,00 & 88.669,72 & 95.590,28 \\ 
  50 a 54 anos & Feminino & 91.483,00 & 87.057,71 & 95.908,29 \\ 
  50 a 54 anos & Masculino & 78.669,00 & 73.776,45 & 83.561,55 \\ 
  55 a 59 anos & Feminino & 74.973,00 & 70.754,33 & 79.191,67 \\ 
  55 a 59 anos & Masculino & 60.808,00 & 57.156,19 & 64.459,81 \\ 
  60 a 64 anos & Feminino & 58.291,00 & 54.712,43 & 61.869,57 \\ 
  60 a 64 anos & Masculino & 45.354,00 & 43.208,53 & 47.499,47 \\ 
  65 a 69 anos & Feminino & 44.177,00 & 41.491,08 & 46.862,92 \\ 
  65 a 69 anos & Masculino & 33.157,00 & 30.881,30 & 35.432,70 \\ 
  70 a 74 anos & Feminino & 30.557,00 & 28.264,83 & 32.849,17 \\ 
  70 a 74 anos & Masculino & 21.855,00 & 20.319,54 & 23.390,46 \\ 
  75 a 79 anos & Feminino & 20.222,00 & 18.371,93 & 22.072,07 \\ 
  75 a 79 anos & Masculino & 13.985,00 & 12.338,57 & 15.631,43 \\ 
  80 a 84 anos & Feminino & 12.318,00 & 11.013,56 & 13.622,44 \\ 
  80 a 84 anos & Masculino & 8.251,00 & 6.890,00 & 9.612,00 \\ 
  Mais de 85 anos & Feminino & 9.915,00 & 8.594,39 & 11.235,61 \\ 
  Mais de 85 anos & Masculino & 4.935,00 & 3.907,00 & 5.963,00 \\ 
   \hline
\end{tabular}
\caption{Pirâmide etária} 
\label{tab:piramide}
\end{table}

\clearpage

\subsection{Salários}\label{salarios}

No que diz respeito aos salários, sua distribuição por faixas de salário
mínimo\footnote{O salário considerado foi de R\$ 954,00} é apresentada
na Figura \ref{fig:salarios}.

\begin{figure}[H]
\includegraphics{relatório_files/figure-latex/salarios-1} \caption{Salarios por faixa de SM \label{fig:salarios}}\label{fig:salarios}
\end{figure}

Os dados podem ser consultados na Tabela \ref{tab:salarios}.

\begin{table}[!hb]
\centering
\begin{tabular}{lrrr}
  \hline
faixas\_salario & n & n\_low & n\_upp \\ 
  \hline
Até 1 salário & 149.808,34 & 143.584,27 & 156.032,41 \\ 
  Mais de 1 até 2 salários & 307.819,42 & 300.403,61 & 315.235,23 \\ 
  Mais de 2 até 4 salários & 185.752,94 & 178.620,43 & 192.885,44 \\ 
  Mais de 4 até 10 salários & 166.799,10 & 160.089,33 & 173.508,86 \\ 
  Mais de 10 até 20 salários & 56.617,22 & 52.190,52 & 61.043,92 \\ 
  Mais de 20 salários & 15.292,09 & 13.332,94 & 17.251,24 \\ 
   \hline
\end{tabular}
\caption{Salarios por faixa de SM} 
\label{tab:salarios}
\end{table}

\clearpage
\pagebreak

\subsection{Esgotamento}\label{esgotamento}

Por fim, o esgotamento sanitário é apresentado na Figura
\ref{fig:esgotamento}

\begin{figure}[H]
\includegraphics{relatório_files/figure-latex/esgotamento-1} \caption{Esgotamento sanitário \label{fig:esgotamento}}\label{fig:esgotamento}
\end{figure}

Os números podem ser consultados na Tabela \ref{tab:esgotamento}.

\begin{table}[!hb]
\centering
\begin{tabular}{lrrr}
  \hline
esgotamento\_caesb & n & n\_low & n\_upp \\ 
  \hline
Com Rede Geral (Caesb) & 0,93 & 0,93 & 0,93 \\ 
  Sem Rede Geral (Caesb) & 0,07 & 0,07 & 0,07 \\ 
   \hline
\end{tabular}
\caption{Esgotamento sanitário} 
\label{tab:esgotamento}
\end{table}


\end{document}
